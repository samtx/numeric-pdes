\documentclass[11pt]{article}

\hoffset        2mm 
\voffset        -15mm
\oddsidemargin  0mm
\topmargin      0.5in
\textwidth      6in
\textheight     9in

\usepackage{graphicx}

\begin{document}

\begin{titlepage}

\vspace*{55mm}
\begin{center}
{\huge MATH610-600}\\[1cm]
{\em \huge Programming Assignment \#1}\\[70mm]
{\large \today} \\[15mm]
\end{center}

\begin{flushright}
{\LARGE Sam Friedman}
\end{flushright}

\vfill

\end{titlepage}

\newpage
\section{Problem Specifications}
In this section you should describe the problems in this particular assignment.
I would use a separate subsection for describing each problem.
\subsection{Problem 1 (Deflection of a uniformly loaded plate)}
Here we describe what the first problem in this assignment asks us to do.
\subsection{Problem 2 (...)}
Here we describe what the second problem in this assignment asks us to do.

\section{Preliminaries}
In this section you should describe your approach to solving the problems
in this example. You do NOT need to include the source code for your 
programs in your report. If you feel that you should include some small
parts of the code in order to explain things better, you may do it like
this:
\begin{verbatim}
//This function calculates the factorial (n!) of an integer n>=0
//using a 'for' cycle
unsigned int factorial(unsigned int n)
{
  unsigned int i, result = 1;
  
  for(i=1; i<=n; i++)
    result *= i;

  return result;
}
\end{verbatim}

\newpage

\section{Problem 1(...)}
Give the results and a small discussion of the results that you obtained.

{\bf NOTE:} If you prefer, you may do so in 
a separate subsection for each particular problem.

\section{Problem 2 (...)}
Give the results and a small discussion of the results that you obtained.

\end{document}








