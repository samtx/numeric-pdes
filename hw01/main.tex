\documentclass[12pt]{article}
 \usepackage[margin=1in]{geometry} 
\usepackage{amsmath,amsthm,amssymb,amsfonts}
\usepackage{pgfplots, pgfplotstable}
\pgfplotsset{compat=1.12}
 
\newcommand{\N}{\mathbb{N}}
\newcommand{\Z}{\mathbb{Z}}
 
\newenvironment{problem}[2][Problem]{\begin{trivlist}
\item[\hskip \labelsep {\bfseries #1}\hskip \labelsep {\bfseries #2.}]}{\end{trivlist}}
%If you want to title your bold things something different just make another thing exactly like this but replace "problem" with the name of the thing you want, like theorem or lemma or whatever
 
\begin{document}
 
%\renewcommand{\qedsymbol}{\filledbox}
%Good resources for looking up how to do stuff:
%Binary operators: http://www.access2science.com/latex/Binary.html
%General help: http://en.wikibooks.org/wiki/LaTeX/Mathematics
%Or just google stuff
 
\title{Homework template}
\author{Sam Friedman}
\maketitle
 
 \newpage
\begin{problem}{1.1}
\(-ku^{\prime\prime}(x) + \beta u^\prime(x) = f(x); \ 0 < x < 1; \ u(0) = a;\  u(1) = b:\)

\begin{center}
\begin{tikzpicture}
\begin{axis}[
	scale only axis=true, % The height and width argument only apply to the actual axis
	width=\hsize*.85,
    height=\vsize*.30,
%     grid=both,
    no markers,
	title={Problem 1.1, $\beta = 1$, Solution},
    legend pos=north west,
    xlabel={$x$},
    ylabel={$y$}
]
\addplot table[x=x,y=uextbe1]{solnpr1pt1n200.dat};
\addplot table[x=x,y=uapxbe1]{solnpr1pt1n25.dat};
\addplot table[x=x,y=uapxbe1]{solnpr1pt1n50.dat};
\addplot table[x=x,y=uapxbe1]{solnpr1pt1n100.dat};
\addplot table[x=x,y=uapxbe1]{solnpr1pt1n200.dat};
\legend{Exact,$n=25$,$n=50$,$n=100$,$n=200$}
\end{axis}
\end{tikzpicture}


\begin{tikzpicture}
\begin{axis}[
	scale only axis, % The height and width argument only apply to the actual axis
	width=\textwidth*.85,
    height=\vsize*.30,
	title={Problem 1.1, $\beta = 1$, Error},
%     grid=both,
    no markers,
    legend pos=north west,
    xlabel={$x$},
    ylabel={$y$}
]
\addplot table[x=x,y=ebe1]{solnpr1pt1n25.dat};
\addplot table[x=x,y=ebe1]{solnpr1pt1n50.dat};
\addplot table[x=x,y=ebe1]{solnpr1pt1n100.dat};
\addplot table[x=x,y=ebe1]{solnpr1pt1n200.dat};
\legend{$n=25$,$n=50$,$n=100$,$n=200$}
\end{axis}
\end{tikzpicture}

\begin{table}[h!]
  \begin{center}
    \caption{Autogenerated table from .dat file.}
    \label{table1}
    \pgfplotstabletypeset[
    	columns={n,L2be1,Linfbe1},
		columns/L2be1/.style={column name=$L_2$},
        columns/Linfbe1/.style={column name=$L_\infty$},
%       	columns/L2be8/.style={column name=$L_2$},
%         columns/Linfbe8/.style={column name=$L_\infty$},
%         columns/L2be64/.style={column name=$L_2$},
%         columns/Linfbe64/.style={column name=$L_\infty$}
    ]{nrmspr1pt1.dat} % filename/path to file
  \end{center}
\end{table}
\end{center}
\end{problem}
 
% \begin{proof}
% Proof goes here. Repeat as needed
% \end{proof}

\end{document}